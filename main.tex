% Created 2017-09-01 Fri 20:48
% Intended LaTeX compiler: pdflatex
\documentclass[11pt]{article}
\usepackage[utf8]{inputenc}
\usepackage[T1]{fontenc}
\usepackage{graphicx}
\usepackage{grffile}
\usepackage{longtable}
\usepackage{wrapfig}
\usepackage{rotating}
\usepackage[normalem]{ulem}
\usepackage{amsmath}
\usepackage{textcomp}
\usepackage{amssymb}
\usepackage{capt-of}
\usepackage{hyperref}
\usepackage{minted}
\date{}
\title{to be defined}
\hypersetup{
 pdfauthor={Willian Ver Valen Paiva},
 pdftitle={to be defined},
 pdfkeywords={},
 pdfsubject={},
 pdfcreator={Emacs 26.0.50 (Org mode 9.0.10)}, 
 pdflang={English}}
\begin{document}

\maketitle

\section{Context}
\label{sec:orgb618ff1}
The beginning of XXIth century sees a drastic increase of applications coming
as an assistance to numerous human activities, among which medicine is one of
the main domains. And the use of technology for the healthcare system is the
focus today, with technologies ranging from robots to assist in surgery to
gadgets that measure sugar levels and help diabetic people.

Among those areas, an important one deals with pain, which is a complex
subject to master. The main objectives of the research project, which consists 
in being able to properly measure pain levels, is of a great importance, as it
will help medical doctors to provide a precise treatment especially for
patients suffering from chronic pain condition. It will also allow
pharmaceutical companies to have a better feedback from their clients for the
development of treatments, whether they are pharmaceutical or not.  

But pain assessment is not a simple task and has been a long lasting challenge even for
human evaluators. Many scales and methods have been created to tackle such a problem
\cite{wong1996wong,mccaffery1999pain,portenoy1996visual,melzack1975mcgill,galer1997development,gracely1988descriptor}
with limited efficiency. So far, it is still difficult to  have an accurate
assessment of pain, all the more important when it comes to deal with chronic
pain.  

As a matter of fact, as pain is a subjective feeling, the most used
measurement is the auto evaluation of the patient, but even if it gives fairly acceptable
results, it still falls short on some aspects, notably when it comes to simulated pain
\cite{gwen2007faces}, or, most commonly, when the patient has difficulties to
communicate, as is the case of infants and some elderly people
\cite{lucey2011automatically}. 

And the creation of a method capable of providing a precise assessment for the level of pain
felt by patients could help the present diagnosis system overcome a complicated
barrier. 


\subsection{State of the art}
\label{sec:orge104ed8}
Up to now, some serious research has been done to achieve a reliable pain
detection via facial expressions. To reach this objective, the systems used datasets of images
labeled with FACS codes (\emph{Facial Action Coding System}) \cite{lucey2011painful}. 

The main work done on the subject was the research carried at the  Pittsburgh
University \emph{"Automatically Detecting Pain in Video Through Facial Action
Units"} \cite{lucey2011automatically}. This research was mainly based on the
use of the FACS to detect pain on a given frame, but obtained elementary
results such as \emph{"pain or no pain"}.  

Another research, done on the same subject with the same dataset, is the
research carried at the university of Aalborg \cite{bellantonio2016spatio}.
This more recent research did achieve more promising results by using a
combination of convolutional neural networks, and recurrent neural networks.
By using the combination of these two methods, it was possible to consider
not uniquely the one frame and FACS, but also to analyse the previous frames
of the considered sequence, and to get a better precision for the result
achieving a scale between \emph{"no pain"} , \emph{"weak pain"} or \emph{"strong pain"} what
is significantly better.  

\subsection{Methodological approach}
\label{sec:orgc28aa75}
This thesis proposes a new approach to the problem by using a new dataset
generated with the contribution of volunteering people who suffer from
chronic pain. While a new dataset is a good start, train a deep learning
model can take large amounts of data.  

But on the other side we can rely on the practioners who have a long experience and
knowledge of the problem, and should not be excluded from the scope.
A practioner uses more than just an image to assess pain, they take many other
factors into consideration. This is why the project proposes to tap the
practioners' experience to compensate the small amount of training data, what
could be compared to the functionning of expert systems
\cite{giarratano1998expert}.

However it is not clear up to this day how it will be possible to reconcile deep
learning and expert systems. The aim of this thesis will be to explore ways
to achieve this.  


\section{Objectives}
\label{sec:org64d7903}
\subsection{Expected results}
\label{sec:org9fb423b}

\subsection{Scientific and technological challenges}
\label{sec:orgd5ce104}

\section{Organization}
\label{sec:orga849b58}
\subsection{General organization}
\label{sec:org1d18b0c}

\subsection{Planning}
\label{sec:orgbf4df68}



\bibliographystyle{unsrt}
\bibliography{repport}
\end{document}